\documentclass[11pt,fleqn]{article} % Default font size and left-justified equations

%\usepackage{standalone}

\usepackage{todonotes}
\usepackage{color}
% use \todo{note} OR \missingfigure{Add my picture here}

\include{structure}

\setcounter{secnumdepth}{5}
\setcounter{tocdepth}{5}
 
%\usepackage{arabtex}

\usepackage{verbatim}

\raggedbottom

\begin{document}

%define macros for commonly used terms that require special formatting
\newcommand \sscope {\textit{SectorScope}\xspace}
\newcommand \hdb {\textit{hashdb}\xspace}
\newcommand \aut {\textit{Autopsy}\xspace}
\newcommand \bulk {\textbf{bulk\_extractor}\xspace}
\newcommand \mdd {\textbf{md5deep}\xspace}
\newcommand \fiwalk {\textbf{fiwalk}\xspace}

\hypersetup{%
    pdfborder = {0 0 0}
}

\lstdefinestyle{customfile}{
basicstyle=\footnotesize\ttfamily, frame=single, float=htpb}

\input{./title.tex}

\pagenumbering{roman}
\setlength{\parindent}{0pt} %remove indenting from whole document
\newpage
\thispagestyle{empty}
\mbox{}
\newpage

\tableofcontents
\newpage
\pagenumbering{arabic}
\newpage

\section{Introduction}
\subsection {Overview of \sscope}
\sscope is a graphical interface tool for viewing blocks that match blacklist block hashes stored in a \hdb database. \sscope includes interfaces for scanning media images, ingesting source files into \hdb databases, and directly viewing raw media image bytes.\\

\subsection{Obtaining \sscope}
\label{Obtaining}
\sscope requires \hdb. Both tools are readily available for Windows systems, Linux flavors, and MacOS.  A Windows installer is available for Windows users.  A source code distribution is available for Linux and Mac users.  Developers may download these tools directly from source available on GitHub.\\

\subsubsection{Installing on Windows}
Download and run the latest \sscope Windows installer available under \url{https://github.com/NPS-DEEP/NPS-SectorScope/blob/master/build}. Click on the latest \verb+.exe+ file, click on \verb+View Raw+, save it, and then run that.\\

\hdb is available under \url{http://digitalcorpora.org/downloads/hashdb}. Please see \hdb documentation for further information on installing \hdb.\\

\subsubsection{Installing on Linux or Mac}
Download and unzip \sscope files from the \verb+.zip+ file under \url{https://github.com/NPS-DEEP/NPS-SectorScope/blob/master/build}.\\ Please see \url{https://github.com/NPS-DEEP/hashdb/wiki/Installing-hashdb} for information on installing \hdb.\\

\section{Managing Massive Datasets}
Block hash scans can produce a large amount of matches. \sscope provides filters to highlight and ignore data:
\begin{itemize}
\item \textbf{Highlight Sectors and Hashes}\\
Entire sectors or individual hashes may be highlighted.
\item \textbf{Ignore Sectors and Hashes}\\
Entire sectors or individual hashes may be ignored.
\item \textbf{Ignore Entropy Range}\\
All hashes from blocks below or above given entropy values may be ignored.
\item \textbf{Ignore Maximum Duplicate Hashes}\\
All hashes with a duplicates count above a maximum may be ignored.
\item \textbf{Ignore Auto-filter Labeled Hashes}\\
All hashes marked with a block label may be ignored.
\end{itemize}
\sscope is able to provide filtering by using block and source information stored in the \hdb database. Specifically:
\begin{itemize}
\item \textbf{Block Entropy}\\
Low entropy blocks are frequently nonprobative.
\item \textbf{Block Label}\\
\hdb generates block labels for blocks possessing specific characteristics. \hdb generates several block labels. Defining effective block labels is an area of research.
\item \textbf{Source Size}\\
The source size is used to calculate the percentage of a source that was matched.
\item \textbf{Count}\\
The count of matches for a block and the sub-count of matches contributed by each source for a block are used to indicate how common a block is.
\item \textbf{Source Label}\\
Source labels may be used to infer expected entropy values. \hdb and \sscope do not use this information. This is an area of research.
\end{itemize}

\section{Using \sscope}
\subsection{Starting \sscope}
\begin{itemize}
\item Type \verb+sectorscope.py+. Type \verb+sectorscope.py -h+ for help on options.
\item In \aut, click the \verb+SectorScope+ property.
\end{itemize}

\subsection{Main Window}
Here is an example screenshot of the \sscope main window:\\
\includegraphics[scale=.4]{screenshots/main_window}\\

\subsection{Menu Controls}
Here are the menu controls:\\
\includegraphics[scale=.4]{screenshots/menu_controls}\\

\subsubsection{Open Scan File}
Here is the Open Scan File window:\\
\includegraphics[scale=.4]{screenshots/open_scan_file}\\
\begin{itemize}
\item \textbf{Scan File}\\
When \hdb performs a scan, all matches are placed in this scan file. The scan file also contains the path to the media image scanned and the hash database scanned against. If these paths move, alternate paths may be provided.
\item \textbf{Alternate Media Image}\\
An alternate path to the media image to use if the path in the scan file is incorrect. Keep blank to use the default.
\item \textbf{Alternate Hash Database}\\
An alternate path to the hash database to use if the path in the scan file is incorrect. Keep blank to use the default.
\item \textbf{Sector Size}\\
The sector size for this media, typically 512 bytes. \sscope will display sectors of this size. \sscope will zoom in to this size.
\end{itemize}

\subsubsection{Scan Statistics}
Here is an example of scan statistics:\\
\includegraphics[scale=.4]{screenshots/scan_statistics}\\
Statistics for the opened scan include the path to the scan file, media image, and hash database, the image size, block size, and number of matched paths, hashes, and sources.

\subsubsection{Ingest}
Here is the \sscope Ingest window:\\
\includegraphics[scale=.4]{screenshots/ingest}\\
Use this window to ingest files or media images into a hash database.
\begin{itemize}
\item \textbf{Source Directory}\\
The path to the source files to ingest recursively from.
\item \textbf{Hash Database}\\
The path to the hash database to import block hashes into.
\item \textbf{Repository Name}\\
The repository name to associate the imported sources with. Leave blank to use the default, which is the source directory path.
\item \textbf{Step Size}\\
The step size to move along while calculating block hashes. The byte alignment must be compatible with the step size, specifically, byte alignment must be divisible by step size.
\item \textbf{Make New Database}\\
If this is selected, a new database will be created instead of using an existing databases. The new database will be created with the following:
  \begin{itemize}
  \item \textbf{Block Size}\\
  The block size to use for calculating the block hash.
  \item \textbf{Byte Alignment}\\
  An optimization parameter, typically the smaller of the block size and the step size.
  \end{itemize}
\end{itemize}

\subsubsection{Scan}
Here is the \sscope Scan Media Image window:\\
\includegraphics[scale=.4]{screenshots/scan}\\
Use this window to scan a media image for matching block hashes.
\begin{itemize}
\item \textbf{Media Image}\\
The path to the media image to scan.
\item \textbf{Hash Database}\\
The path to the hash database to scan against.
\item \textbf{Sector Size}\\
The sector size for this media, typically 512 bytes. \sscope will display sectors of this size. \sscope will zoom in to this size.
\end{itemize}

\subsubsection{Information}
This button opens a window showing the version of \sscope and \hdb.

\subsection{Highlight and Ignore}
Here is an example view of the highlight and ignore controls:\\
\includegraphics[scale=.4]{screenshots/highlight_and_ignore}\\
These inputs control which hashes and sources are highlighted or ignored in the histogram bar and in the source list. Highlighted items are shown in green. Ignored items are not displayed.
\subsubsection{Highlight}
\begin{itemize}
\item \textbf{+H}\\
Select a range and highlight hashes in this range.
\item \textbf{+S}\\
 Select a range and highlight all sources containing hashes in this range.
\item \textbf{-H}\\
Un-highlight all highlighted hashes.
\item \textbf{-S}\\
Un-highlight all sources.
\end{itemize}

\subsubsection{Ignore}
\begin{itemize}
\item \textbf{+H}\\
Select a range and ignore hashes in this range.
\item \textbf{+S}\\
Select a range and ignore all sources containing hashes in this range.
\item \textbf{-H}\\
Stop ignoring all ignored hashes.
\item \textbf{-S}\\
Stop ignoring all ignored sources.
\item \textbf{Entropy Below}\\
Ignore hashes with an entropy value below this threshold.
\item \textbf{Entropy Above}\\
Ignore hashes with an entropy value above this threshold.
\item \textbf{Max Duplicate Hashes}\\
Ignore hashes matched more than a maximum number of times.
\item \textbf{Auto-filter}\\
Ignore hashes if they have been flagged as potentially non-probative based on experimental entropy calculations.
\end{itemize}

\subsection{Histogram}
Here is an example view of the histogram portion:\\
\includegraphics[scale=.4]{screenshots/histogram}\\

\subsubsection{Graph View}
The histogram graph area shows the frequency of hash matches at given offsets in the media image. Bar width depends on the zoom level. Bar color indicates hashes present within the region of the bar. Blue indicates hashes present. A blue tick above the bar indicates total hashes when hashes are ignored. A green bar indicates highlighted hashes.\\

\subsubsection{Graph Annotation}
\begin{itemize}
\item \textbf{Bar width}\\
The number of sectors or bytes spanned by each bar, depending on the offset format.
\item \textbf{Bar matches}\\
The number of matches under the bar that the cursor is hovering over.  Three values are shown: total matches, highlighted matches (h), and ignored matches (i).
\item \textbf{Range}\\
The amount of the media image that the selected range spans.
\item \textbf{Bounds}\\
The first and last offset across the histogram is shown.
\item \textbf{Y-axis scale}\\
The horizontal scale, indicating the number of hash matches required to reach that height. This scale may be toggled between auto-scaling and no scaling.
\end{itemize}

\subsubsection{Graph Button Controls}
\begin{itemize}
\item \textbf{Zoom to full scale}\\
Zoom out so the view spans the entire media iamge.
\item \textbf{Zoom to range selection}\\
Zoom in on the selected range to fit the view.
\item \textbf{Show Hex View}\\
Open a window to show the hexadecimal bytes of the block under the cursor.
The following example hex view shows bytes for matched block hash \verb+0x3982...+ at media image offset \verb+0x09c40000+.\\
\includegraphics[scale=.4]{screenshots/hex_view}\\
\item \textbf{Manage annotations}\\
Open a window to enable or disable annotation categories. Here is an example window showing that annotations from disk partitions and file system sectors are enabled:\\
\includegraphics[scale=.4]{screenshots/manage_annotations}\\
  \begin{itemize}
  \item Disk partition annotations are obtained by running the TSK mmls command.
  \item File system sectors are obtained by running the TSK fsstat command.
  \end{itemize}
New annotation types may be added in the future. \sscope accepts annotations in the form of \verb+type+, \verb+offset+, \verb+length+, and \verb+text+.
\item \textbf{Toggle Offset Units}\\
Toggle the displayed offset units between sectors, decimal bytes, and hexadecimal bytes.
\item \textbf{Toggle auto-y-axis}\\
Enable or disable auto-y-axis histogram bar scaling.
\end{itemize}

\subsubsection{Graph Cursor Controls}
Cursor movement, click, drag, and wheel motions manipulate the graph view:
\begin{itemize}
\item Move the mouse to position the cursor. Enable hex view to see the bytes under the cursor.
\item Drag the mouse to select a range. Zoom to expand the graph to the selected range. Ignore or highlight hashes or sources in this selected range.
\item Right-click drag to pan the graph.
\item Roll the wheel to zoom in and out.
\end{itemize}

\subsection{Media Image Annotations}
Here is an example view of media image annotations:\\
\includegraphics[scale=.4]{screenshots/annotations}\\
Disk partitions and file system sectors are shown. They run over the top of each other because the histogram view is fully zoomed out and there are so many annotations present. To help preserve readability, only annotations that refer to a range that spans a whole histogram bar or more are shown in black, annotations that span less than one histogram bar in width are shown in gray, and annotations that span less than one tenth of a histogram bar are shown in light gray.

\subsection{Source Table}
Here is an example view of the source table:\\
\includegraphics[scale=.4]{screenshots/source_table}\\

\subsubsection{Source Table Contents}
Rows in the source table contain the following information:
\begin{itemize}
\item \textbf{Source Hash}\\
The hash hexcode of the source file.
\item \textbf{\%Match}\\
The percent of this source that was matched by the scan.
\item \textbf{\#Match}\\
The number of matches matched by the scan.
\item \textbf{\#M(h)}\\
The number of matches that are highlighted.
\item \textbf{Size}\\
The size of the source file matched.
\item \textbf{Repository Name}\\
A repository name associated with this source file.
\item \textbf{Filename}\\
A filename associated with this source file.
\end{itemize}

\subsubsection{Source Table Controls}
\begin{itemize}
\item If a range is selected, only sources with hashes in that range are shown. Otherwise, all hashes are shown.
\item If all hashes of a source are ignored, that source will not be shown.
\item Click on a source to highlight it. It will show up green. Portions of the histogram bars contributed  by that source will show up green. Re-click to un-highlight.
\item Left-click on a source to ignore it. It will be removed from the sources table. Portions of the histogram bars contributed  by that source will be removed. Ignored sources may be un-ignored using the \textbf{-S} Ignore control.
\end{itemize}

\section{Example}
In this example, we create a blacklist database of block hashes, then scan a media image for matches:

\begin{enumerate}
\item Identify a directory containing your blacklist source data.
In this example, we use the Kitty Material demo source files available at \url{http://digitalcorpora.org/corpora/scenarios/2009-m57-patents/KittyMaterial} and scan for matches in the demo media image available at \url{http://digitalcorpora.org/corpora/scenarios/2009-m57-patents/drives-redacted/jo-favorites-usb-2009-12-11.E01} which contains blacklist block hashes from the Kitty demo:
\item Start \sscope by typing the following at a command prompt:
\begin{Verbatim}[commandchars=\\\{\}]
\verbbf{sectorscope.py}
\end{Verbatim} 
\item Click on the Ingest menu icon 
\includegraphics[scale=.4]{screenshots/ingest_menu_icon}
to open the \sscope Ingest window.
\item Fill in the fields:
  \begin{itemize}
  \item Set Source Directory to the directory containing your blacklist source data.
  \item Set Hash Database to your new hash database.
  \item Set the Repository Name to the name of this case dataset, or leave blank to use the source directory as the repository name.
  \item Use the default step size to ingest along sector intervals.
  \item Keep Make New Database checked since you will not be ingesting block hashes into an existing database.
  \item Use the default block size to calculate sector-sized block hashes.
  \item Use the default byte alignment since the blocks are sector-aligned.
  \end{itemize}
\item Click the \verb+Start+ button to begin the process of ingesting block hashes from files under the source directory. Progress information will be displayed. When done, status will indicate \verb+Done+. Close the window.
\item Click on the scan media image icon
\includegraphics[scale=.4]{screenshots/scan_media_image_icon}
to open the \sscope Scan Media Image window.
\item Fill in the fields:
  \begin{itemize}
  \item Set the path to the media image to be scanned.
  \item Set the path to the newly created hash database.
  \item Set the path to the scan output file that will be generated.
  \item Use the default step size to scan along sector intervals.
  \end{itemize}
\item Click the \verb+Start+ button to begin the process of scanning for matching block hashes. Progress information will be displayed. When done, status will indicate \verb+Done+. Close the window.
\item Click on the Open scanned output icon
\includegraphics[scale=.4]{screenshots/open_scanned_output_icon}
to open the Open Scan File window which opens the scan data into \sscope.
\item Fill in the fields:
  \begin{itemize}
  \item Set the path to newly created scan file.
  \item Leave the alternate media image field blank since the path to the media image defined in the scan file is correct.
  \item Leave the alternate hash database field blank since the path to the hash database defined in the scan file is correct.
  \item Use the default sector size to view sector offsets for 512-byte-sized sectors.
  \end{itemize}
\item Click the \verb+Open+ button to load this scan dataset into \sscope. It can take 30 seconds or more to load large datasets. \sscope currently does not show progress during open. Close the window.
\item Manipulate \sscope controls to examine matched data.
\end{enumerate}

\end{document}
